% 
% installing.tex : part of the Mace toolkit for building distributed systems
% 
% Copyright (c) 2011, Charles Killian, Dejan Kostic, Ryan Braud, James W. Anderson, John Fisher-Ogden, Calvin Hubble, Duy Nguyen, Justin Burke, David Oppenheimer, Amin Vahdat, Adolfo Rodriguez, Sooraj Bhat
% All rights reserved.
% 
% Redistribution and use in source and binary forms, with or without
% modification, are permitted provided that the following conditions are met:
% 
%    * Redistributions of source code must retain the above copyright
%      notice, this list of conditions and the following disclaimer.
%    * Redistributions in binary form must reproduce the above copyright
%      notice, this list of conditions and the following disclaimer in the
%      documentation and/or other materials provided with the distribution.
%    * Neither the names of the contributors, nor their associated universities 
%      or organizations may be used to endorse or promote products derived from
%      this software without specific prior written permission.
% 
% THIS SOFTWARE IS PROVIDED BY THE COPYRIGHT HOLDERS AND CONTRIBUTORS "AS IS"
% AND ANY EXPRESS OR IMPLIED WARRANTIES, INCLUDING, BUT NOT LIMITED TO, THE
% IMPLIED WARRANTIES OF MERCHANTABILITY AND FITNESS FOR A PARTICULAR PURPOSE ARE
% DISCLAIMED. IN NO EVENT SHALL THE COPYRIGHT OWNER OR CONTRIBUTORS BE LIABLE
% FOR ANY DIRECT, INDIRECT, INCIDENTAL, SPECIAL, EXEMPLARY, OR CONSEQUENTIAL
% DAMAGES (INCLUDING, BUT NOT LIMITED TO, PROCUREMENT OF SUBSTITUTE GOODS OR
% SERVICES; LOSS OF USE, DATA, OR PROFITS; OR BUSINESS INTERRUPTION) HOWEVER
% CAUSED AND ON ANY THEORY OF LIABILITY, WHETHER IN CONTRACT, STRICT LIABILITY,
% OR TORT (INCLUDING NEGLIGENCE OR OTHERWISE) ARISING IN ANY WAY OUT OF THE
% USE OF THIS SOFTWARE, EVEN IF ADVISED OF THE POSSIBILITY OF SUCH DAMAGE.
% 
% ----END-OF-LEGAL-STUFF----
\section{Getting and Installing Mace}
\label{sec:installing}

\subsection{Mace dependencies}

Mace requires the following system software packages:

\begin{description}

\item[gcc/g++] The GNU C and C++ compilers.  Mace is tested with GCC
  versions 3.4 (including MinGW), 4.0, 4.1, and 4.2.  It has in the past
  worked with a few GCC version 3.3.X systems, but most frequently
  causes internal compiler segmentation faults.  Using at least version
  3.4 is highly recommended.  This should also be accompanied with a
  build system that cmake can output builds for.

\item[perl] Perl version 5.8 or greater.  Additionally,
Class-MakeMethods and Parse-RecDescent modules are needed (and are
included in mace-extras).
%TODO: I don't think people need to get these anymore, with mace-extras,
%right?

\item[system libraries] libpthread, libm, libcrypto, libstdc++, and libssl.

\item[libboost] Used in a few places for shared pointers and lexical casting.

\item[libdb\_cxx] Version 4.2, 4.3 or 4.4.  This is the Berkeley DB
  package, and is optional for building Mace.

\item[cmake] Mace uses the \href{http://www.cmake.org}{CMake} build
  system for cross-platform building.  

\item[lgrind] For making the documentation from the source files, beautifying 
  the source code.  Also needs LaTeX.

\end{description}

Mace is regularly tested on Debian GNU/Linux and CentOS.  It has been
tested on 32 and 64 bit Linux systems, 64 bit OSX systems, and WinXP
(under MinGW). 

\paragraph{Compiling on Windows.}
When building Mace under MinGW (Windows), users will have to install
OpenSSL libraries, GNU libregex, pthreads for windows, and (if desired)
build the Berkeley DB C++ library by hand (the binary distribution does
not include it as far as we could tell).

% XXX
% Chip, what else should we write here about platforms?


\subsection{Getting the Mace source code}
\label{sec:downloading}

Mace is Open Source Software which is be published under a BSD-style license.
The latest Mace source code release can be found at the following URL: \\

\href{http://mace.ucsd.edu/release}{http://mace.ucsd.edu/release} \\

Download this file and save it in the directory where you will be
doing your development.  Unpack it using 
\command{tar -xvzf mace-latest.tar.gz}
where  latest is replaced by the version string
from the version downloaded.

Or download mace from the subversion repository using: \\

\command{svn co http://mace.ucsd.edu/svn/mace/trunk mace}

\subsection{Installing Mace}
\label{sec:unpacking}

Currently, the Mace distribution does not require the installation of any
Mace-related files outside of the Mace source tree.  All development using Mace
can be done inside the source tree.

To build the Mace distribution (in brief), use the following commands:

\begin{screen}
\command{
$ cd mace
$ mkdir builds
$ cd builds
$ cmake ../
$ make edit\_cache #or ccmake .
$ make
}
\end{screen}
%$

This will build the Mace libraries (described in
\S~\ref{sec:lib}), services
(\S~\ref{sec:packaged-services}), and applications
(\S~\ref{sec:applications}).

\subsection{The Mace Build System}
\label{sec:cmake}

Mace uses CMake to generate its build system.  CMake greatly simplified
the process of supporting various platforms, with different paths,
operating systems, bit sizes, and build systems.  It can even generate
visual studio projects (though Mace cannot build presently under visual
studio).  

This document is far from an instruction guide to using CMake, but in
this section, we'll cover several topics about using CMake to build
Mace.  If you know CMake, or just want to get on to learning about Mace,
you can skip the rest of this section for now.

\subsubsection{CMake is a build system generator}

CMake does not actually build Mace.  What it does is generate the build
system, which in turn does the actual work of building Mace.  In my own
work with Mace, this output build system has always been Makefiles for
use with GNU Make.  Readers should refer to the CMake documentation for
other options.  References following will assume a Makefile build.

\subsubsection{Out of source builds}

Mace now requires builds to be done "out-of-source."  What this means is
that none of the generated code, object files, or application binaries
are placed inside the source directory.  Attempts to configure CMake for
in-source build will be rejected by the Mace CMakeLists.txt (CMake
configuration file).  Out of source builds have the advantage that they
do not modify the original source directory, so building multiple
outputs is simplified, and cleaning up builds is also simpler.  Also, it
avoids confusion from having generated source files along side the
original source files.

\subsubsection{Running CMake}

To run CMake, use \command{cmake [path-to-mace]}, where you supply cmake
the path to the top level Mace directory.  CMake will read the
CMakeLists.txt in that directory, and configure the build accordingly.
You should run cmake from the directory in which you intend to build
Mace.  If all goes well, it will indicate that the build system is
generated (not just configured), and there will now be a Makefile you
can use.

\subsubsection{Configuring a CMake Build (and turning on debugging symbols)}

When you run cmake, a number of things may need configuring.  This may
include the paths to various required or optional libraries, header
files, or include directories.  Alternately, you may wish to set a
release type.  There are (at least) two ways to configure a CMake build.
If the build system hasn't been generated yet, you have to use the
first.  That method is \command{ccmake [path]}, where ccmake is either
the path to a partially configured CMake build (usually '.'), or the
path to the Mace top-level directory (to start a new build directory).
This opens the configuration editor, where you can set variables and
reconfigure the build.  After setting a variable, you must first run
"Configure" (perhaps more than once), and finally "Generate", to
regenerated the build system.  Until you do this, changed values may not
be saved.  This method is probably what will be needed after a failed
CMake configuration (e.g. for a missing library).  

If the build system has already been generated, another way to enter the
configuration editor is the command \command{make edit\_cache}.  This is
the method commonly used for changing configuration parameters after a
complete configuration, such as the release type.  By default, CMake
will generated a normal build, in which no special flags are included.
Other options include "DEBUG" (enable debugging symbols),
"RELWITHDEBINFO" (enable moderate optimizations and debugging symbols),
and "RELEASE" (enable extreme optimizations and not debugging symbols).
After changing the release type, use Configure and Generate to
regenerate the build system.

\subsubsection{Changing the CMake Input Configuration Files}

If you edit any of the CMake input files, the build system will be
automatically regenerated the next time you run "make".  Be careful not
to edit any of the generated input files, as these will just be
overwritten.

\subsubsection{Adding New Source Files}

In writing our CMake input files, rather than listing each source file
manually (and editing the input files when adding new files), we use the
globbing features to have CMake scan for input files.  As a result, when
you add new files, CMake will \emph{not detect} that a change has
occurred.  To give CMake the opportunity to include the newly added
files, run \command{cmake rebuild\_cache}, which will cause the build
system to be reconfigured and generated.  (This also applies if new
files were added by subversion, so be aware you may need to run this in
that case).

\subsubsection{Features of CMake Makefiles}

CMake Makefiles have some automatically built features.  These include:
\begin{description}
\item [progress] Percentages are displayed during builds, which give a
rough idea of how far through the build you are.  (These are rough
only---in my experience they may halt either before or after 100\%.  In
one rare circumstance, I saw it get as high as 500\%.
\item [quiet] By default, the builds are quiet, displaying descriptions
of what is going on.  To see the full commands, include "VERBOSE=1" in
the command, such as \command{make VERBOSE=1}.
\item [parallel build] CMake Makefiles support parallel builds.  To run
make with 4 parallel processes, use \command{make -j4}.  
\item [help] From any level of the Makefile, you can run \command{make
help}, which displays all targets you could make.
\item [edit\_cache] \command{make edit\_cache} invokes the CMake configuration editor.
\item [rebuild\_cache] \command{make rebuild\_cache} re-runs cmake to
regenerate the build system.
\item [selective build] When building any particular application or
service, CMake will only build (or rebuild) those services which are
needed (instead of always building all services).
\item [shared libraries] CMake can build Mace libraries as shared
libraries instead of static libraries, by editing the configuration to
set USE\_SHARED to on.  Note that this is not as well tested, and may
not work in all circumstances.
\end{description}
