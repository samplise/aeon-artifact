% 
% advanced.tex : part of the Mace toolkit for building distributed systems
% 
% Copyright (c) 2011, Charles Killian, Dejan Kostic, Ryan Braud, James W. Anderson, John Fisher-Ogden, Calvin Hubble, Duy Nguyen, Justin Burke, David Oppenheimer, Amin Vahdat, Adolfo Rodriguez, Sooraj Bhat
% All rights reserved.
% 
% Redistribution and use in source and binary forms, with or without
% modification, are permitted provided that the following conditions are met:
% 
%    * Redistributions of source code must retain the above copyright
%      notice, this list of conditions and the following disclaimer.
%    * Redistributions in binary form must reproduce the above copyright
%      notice, this list of conditions and the following disclaimer in the
%      documentation and/or other materials provided with the distribution.
%    * Neither the names of the contributors, nor their associated universities 
%      or organizations may be used to endorse or promote products derived from
%      this software without specific prior written permission.
% 
% THIS SOFTWARE IS PROVIDED BY THE COPYRIGHT HOLDERS AND CONTRIBUTORS "AS IS"
% AND ANY EXPRESS OR IMPLIED WARRANTIES, INCLUDING, BUT NOT LIMITED TO, THE
% IMPLIED WARRANTIES OF MERCHANTABILITY AND FITNESS FOR A PARTICULAR PURPOSE ARE
% DISCLAIMED. IN NO EVENT SHALL THE COPYRIGHT OWNER OR CONTRIBUTORS BE LIABLE
% FOR ANY DIRECT, INDIRECT, INCIDENTAL, SPECIAL, EXEMPLARY, OR CONSEQUENTIAL
% DAMAGES (INCLUDING, BUT NOT LIMITED TO, PROCUREMENT OF SUBSTITUTE GOODS OR
% SERVICES; LOSS OF USE, DATA, OR PROFITS; OR BUSINESS INTERRUPTION) HOWEVER
% CAUSED AND ON ANY THEORY OF LIABILITY, WHETHER IN CONTRACT, STRICT LIABILITY,
% OR TORT (INCLUDING NEGLIGENCE OR OTHERWISE) ARISING IN ANY WAY OUT OF THE
% USE OF THIS SOFTWARE, EVEN IF ADVISED OF THE POSSIBILITY OF SUCH DAMAGE.
% 
% ----END-OF-LEGAL-STUFF----
\section{Advanced Features}
\label{sec:advanced}

% automatic failure detection, uses, provides, auto types, service
% composition, registration ids, authentication....

FUTURE NOTE: This section will eventually be in the context of building a
larger service such as the DHT service.  For now it just describes many of the
different topics which will be covered in context in the future.

\subsection{Registration UIDs}
\label{sec:reguid}

Registration UIDs were seen in action in the detailed ping example.
Registration UIDs represent the logical link between two services (or
more specifically, a service and one or more handlers).  So when a
tree service uses a routing service, the tree service should use a
registration UID to refer to itself when requesting data be sent.
This allows the routing service to know the destination
\classname{ReceiveDataHandler}.  It also allows the hypothetical tree
service receiving the data to know the source service variable, in
case it uses more than one routing service.  The requirement for
registration UIDs is that these registration UIDs be unique, but that
they are the same across the nodes of the network.  The present
implementation allows this by assuming parallel code and a fixed load
order, where the UIDs are assigned.

When registering handlers for a service variable, \literal{-1} may be
passed in as the UID the first time, to which, the assigned
registration UID will be returned.  This registration UID should be
used in all future calls to the same service variable, both for
registrations and normal API calls.  This is handled automatically for
generated services; the service variable itself is a reference to the
registration UID.  No assumptions should be made about the assignment
of UIDs; in the future this may be used like a cookie, and
presentation of the cookie will be a loose form of authentication.

Note: Registration UIDs are also addressed in the FAQ~(\ref{sec:faq}).

\subsection{Service Composition}
\label{sec:servicecomposition}

One of the powerful features of Mace is the ability to leverage services others
have written and to add features to them.  For example, consider the
implementation of a generic tree multicast service.  This service makes use of
two separate services -- one which takes care of maintaining the tree
structure, and the other which manages routing data over the network.  Since
tree services provide a common API to give the tree structure, a single generic
service can implement the algorithm to multicast data to parents and children.

Constructing the services for a generic tree service might look like this:

\begin{programlisting}
services {
  Route data = TcpTransport();
  Tree tree = RandTree();
}
\end{programlisting}

This indicates that the GenericTreeMulticast service will work over whatever
router and tree are passed in, but that if either service is not passed in to
the constructor, new services will be created according to the default listed.

\subsection{Local Addresses}
\label{sec:localaddrs}

Each service exports a \emph{Local Address} to those which use it, in
the form of a \typename{MaceKey}.  For many services and situations, 
this will just be the IPV4 address the node is bound to.  However, 
the network address could include a proxy address (where peers should
send traffic destined for it), or it could be part of a different address
family together.  Services like Chord and Pastry use addresses based on
SHA1 hashes, and systems like SkipNet use free-form strings.

Any service may be asked for its notion of the local address via the
\function{getLocalAddress()} function in \filename{ServiceClass.h}.  Services
should use this address from their lower-level services rather than make
assumptions, so that they may more generically run over any service providing
the right interface.  Note, however, that the addressing a service exposes to a
higher layer may be different from the addressing of lower services (which may
in fact differ from each other).  In this fashion Pastry reports a local
address based on the SHA1 hash of the local IP address, but uses IP addressing
for all internal communication.

For those with a true need to know about the IPV4 address of the local machine,
they may use the helper function \function{Util::getAddr()}.  A somewhat better
helper function to use is \function{Util::getMaceAddr()}, which also returns
the proxy address set.  By default, this does a lookup on the machine's
hostname, and returns the address associated with that hostname and the default
port.  However, for machines which do not have an address associated with their
machine's name, for machines which have multiple addresses and you want to
control which one is used, or if the default code just isn't working for you,
you may set the parameter \variablename{MACE\_LOCAL\_ADDRESS} in the Params
class, which should be in the format ``hoststring:port'' or
``hoststring:port/hoststring:port'', where the first is the locally bound
address and port, and the second appends to that the proxy address (which can
be used in conjunction with port forwarding on firewalls, for example).

\subsection{Auto Types}
\label{sec:autotype}

Auto types are really just inline data structures with compiler generated
features.  These features include Serialization for Messages,
XmlRpc-serialization for interfacing with remote processes, the ability to dump
and print these in automated debugging, and to create ``nodes'' from them to
store and use in \classname{NodeCollections}.  While it is true that you could
define these data structures in external header files, and write your own
\function{serialize}, \function{toString}, and other sundry functions, being
able to include them in your service is a convenience.  Since it is common for
services to have some private internal representation of its state, this
represents a simple way to do so.

To define an auto type, simply include it in the \symbolkw{auto\_types} block,
and define it as you would any other data structure\footnote{Though without the
trailing semicolon}.  If you want to use the type in a
\typename{NodeCollection}, specify that it is a node type in its defaults.  The
full set of options will be described in the technical manual.  For an example,
however, consider the following example:

\begin{programlisting}
auto_types {
  child __attribute((node(fail_detect=enabled; score=delay;))) {
    double delay __attribute((serialize(no)));
  }
}
\end{programlisting}

This defines a child class which represents a node, has auto failure detection
code generated automatically\footnote{Although failure detection is enabled for
this type, no detection will take place unless an instance of a node type or
node collection is specified to perform failure detection using a particular
route service}, and which returns the value of the delay field for the default
score function.  Then, by scoring the round-trip delay in the delay field, you
can quickly query a node collection to find out who has the least score --
therefore the smallest round-trip latency.  Furthermore, the delay field will
not be serialized if a child is sent over the network in the message.  This can
be useful if for example, the type cannot be easily serialized, or if you are
trying to conserve bandwidth.

\subsection{Message Defaults}
\label{sec:msgdefs}

When using messages for communication with other nodes\footnote{And really, what
else are you going to use messages for?}, it may be desirable to utilize
per-message defaults for each call you might make with a message.

Consider this: when developing a new routing protocol, you may want to consider
using multiple TCP services (\emph{i.e.} socket connections or application
queues) for communication in your service.  The point of using multiple TCP
services is that each service will maintain an independent queue of messages to
send your peers.  Therefore, if one is expected to frequently be full due to
congestion backup, it is commonly desirable to have another one on hand which
you will use more sparingly for more important control messages.  You can even
set the queue size independently for each TCP service when it is constructed.
Alternately, you may have two routing services, one of which you expect will be
encrypted, and the other in plain text.  In both of these cases, the chances
are that although you have multiple ways to route messages, you will have one
way that you "normally" send each type of message.  Specifying per-message
defaults allows you to set default parameters for fields following the message.
In the \classname{RouteServiceClass}, this includes the default
\typename{registration\_uid\_t}\footnote{And remember that this is how you
actually specify which lower-level service to route the message over}.  This
way, when you actually send a message, unless you override the defaults, what
you ``normally'' want to happen will.  Furthermore, if you later change your
mind about what you ``normally'' want to do, you can change it in a single
place.

Now, as an example, consider the \function{RouteServiceClass::route()}
call.  Without using defaults, in the transitions of the \mac file,
you would write the following code to send a \typename{foo} message with no
fields to peer ``dest'' over the ``control\_'' routing service:

\begin{programlisting}
downcall_route(dest, foo(), control_);
\end{programlisting}
But, if you specify the following message defaults when defining foo:

\begin{programlisting}
messages {
  foo [downcall_route(const MaceKey&, const Message&, registration_uid_t regId = control_);] { }
}
\end{programlisting}
Then when you actually want to send it you can just write:

\begin{programlisting}
downcall_route(dest, foo());
\end{programlisting}

Apart from being much shorter, if you later add an encrypted route service 
to your service, you can change which service the \typename{foo} message is 
sent over by just changing the message defaults.  And you can still, on 
a case-by-case basis override the defaults.

\subsection{Method Remapping Block}
\label{sec:method_remapping}

Earlier, when writing the Ping service, we glossed quickly over the
\symbolkw{method\_remappings} block, indicating simply that it is
important to make sure that messages are handled properly.  The basic
issue is that since the Mace compiler generates code based on
arbitrary interface header files, it does not know \emph{a priori} what
strings mean to services.  For example, they could in fact be any of
the messages defined in the messages block, or any serializable type,
or just a plain old string.  By default, the compiler assumes a string
is just a string, and relies on the annotations in this block by the
service implementor to tell it otherwise.  An added effect of putting
this annotation in the service description is that different services
may interpret and use the strings differently.  However, in the
future, we may also consider annotating the interface files to allow a
different default behavior.

A second issue is how to set default parameters for calls made from
the service.  Calls made with Message parameters have the ability to
set per-message defaults, as discussed in the previous section
(\S~\ref{sec:msgdefs}).  But we need to provide a mechanism for
setting defaults for non-messaging calls, and to provide defaults for 
all messages.

In the future, we will also be allowing more flexible remappings, which for
example allow you to remap types using anonymous inline functions, and 
which allow you to remap functions to other functions.

The \symbolkw{method\_remappings} block is broken into two parts:
\symbolkw{uses} calls and \symbolkw{implements} functions.

\subsubsection*{Uses Calls}

Uses calls are calls which a service writer will make from their service into
either services which they use (downcalls), or to handlers which are registered
with them (upcalls).  This declarations in the block should match the declarations
in their respective header files, with the following differences.

\begin{description}

\item[function prefix] Each function should be prefixed either by
\symbolkw{downcall\_} or \symbolkw{upcall\_}, appropriate to its kind.

\item[optional string handling] Any string parameter may have a
serialization instruction provided, which can be any type mace knows
how to serialize and deserialize.  The type, and possibly
\symbolkw{const} and/or reference (\symbolkw{\&}) should appear in the
declaration where the string declaration would normally go.  Then, an
arrow pointing to the right (\symbolkw{->}) is used to indicate that
for this function you will pass in the type you specified, but it
should be converted to the type from the original function declaration
before performing the downcall.  The type modifiers will be applied to
the object as it is passed around for serialization or
deserialization, and do not necessarily have to match those from the
actual function declaration.  \footnote{Note that the Mace compiler
will [soon] correctly handle updating your data item if a non-constant
string reference is used (and therefore possibly modified).  If you
don't want to allow that to happen, you can add the constant modifier
to your declaration of your type, which will prevent the Mace compiler
from deserializing the string as the function call returns.}  This is
why in the \function{downcall\_route} specification, the specification
is \typename{[const Message\&]}.  Since in these calls, a temporary
Message is constructed, the gcc compiler will complain if the Message
parameter is non-const.  The Message is passed by reference, to avoid
unnecessary memory copies before serialization, or to ensure the
deserialized string fills in the provided message.  Note that
``Message'' is a special string for serialization, since this will
cause the compiler to both apply per-message defaults, and to handle
determining the message type on deserialization and calling the
appropriate transition with the message delivery specified.

\item[default values] Default values may be specified.  These are done in the
usual fashion by naming the variable and providing a default value for it.
After doing so, the parameter can be left out of the call, and the default will
be provided. 
\end{description}

\begin{programlisting}
method_remappings {
  uses {
    downcall_route(const MaceKey&, const Message& -> const std::string&, registration_uid_t);
    downcall_joinOverlay(const NodeSet&, const std::string& identity=std::string(), registration_uid_t regId = router_);
    upcall_verifyJoinOverlay(const MaceKey&, const std::string&, registration_uid_t regId = activeAuthenticator);
  }
  //... service-implemented calls will be described next
  implements {
    //... be patient for this one
  }
}
\end{programlisting}

In the above example, we're specifying 3 different uses calls with
different instructions.  In the first case, we're indicating that when
we call \function{downcall\_route}, for the second parameter, instead
of passing in a string, we will pass in a \typename{const Message\&},
and that it should be automatically serialized to a string.  Since it
is a constant reference, there will be no consideration for the
deserialization of the string parameter after the call to
\function{downcall\_route}.  The second call specifies that on the
\function{downcall\_joinOverlay} call, the default identity is the
empty string, and the default registration UID is the one assigned to
the \variablename{router\_} service variable (assuming one exists).
The third mainly differs in that it is specifying an upcall, and that
the regId default should be activeAuthenticator, which may be a state
variable rather than a service variable.


\subsubsection*{Implements}

The \symbolkw{implements} sub-block remaps calls which may be made on the
service the writer is writing.  These include both functions of the API defined
by service classes they provide (downcalls), and callback functions/upcalls for
handlers they register with services they use (upcalls).  Their declaration
also closely follows the original declaration from the header file, and in this
case no qualification about the type of call is necessary since there are
sub-blocks for the different types.  Unlike the uses block, the implements
block does not support default values since these calls are incoming
from elsewhere, they will already have the parameters filled in.  However, the
syntax for specifying serialization instructions are the same, save for the
direction of the arrow being reversed (\symbolkw{<-}) to indicate that the
remote caller is passing in a string and it should be deserialized into your
type before your transitions are called.  The same provisos are true about the
\symbolkw{const} and \symbolkw{\&} modifiers on the type, and serialization
after your transition is done.  This will typically be used to specify the
handling of strings in upcalls either as messages or as other serializable
types.

\begin{programlisting}
method_remappings {
  uses {
    //... see the prior section
  }
  implements {
    upcalls {
      deliver(const MaceKey&, const MaceKey&, const Message& <- const std::string&,
              registration_uid_t);
    }
    downcalls {
      //... the same syntax can be used here
    }
  }
}
\end{programlisting}

In this example there is a single callback specification, and when deliver is
called, the 3rd parameter will be handled like a message.

\subsection{Automated Failure Detection}
\label{sec:failure}

Automated failure detection is presently disabled due to incompatible updates
to the compiler, but will be fixed soon.  Synopsis: After enabling failure
detection for a node-auto type, an instance of a state variable (node or node
collection) may have failure detection active.  Liveness probing will
automatically be done to the specified peers, and when failures are detected,
they are signaled to the service via special transitions firing.

\subsection{Authentication}
\label{sec:authentication}

Authentication will be covered when the DHT is included in the manual.
Synopsis: A single authenticator per-group or per-overlay is registered or set
at any given point in time (per-node).  Local authentication is handled by that
active authenticator, and remote authentication goes locally to that
authenticator.  We are adding identity strings to several of our calls to make
this possible.

