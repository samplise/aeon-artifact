% 
% faq.tex : part of the Mace toolkit for building distributed systems
% 
% Copyright (c) 2011, Charles Killian, Dejan Kostic, Ryan Braud, James W. Anderson, John Fisher-Ogden, Calvin Hubble, Duy Nguyen, Justin Burke, David Oppenheimer, Amin Vahdat, Adolfo Rodriguez, Sooraj Bhat
% All rights reserved.
% 
% Redistribution and use in source and binary forms, with or without
% modification, are permitted provided that the following conditions are met:
% 
%    * Redistributions of source code must retain the above copyright
%      notice, this list of conditions and the following disclaimer.
%    * Redistributions in binary form must reproduce the above copyright
%      notice, this list of conditions and the following disclaimer in the
%      documentation and/or other materials provided with the distribution.
%    * Neither the names of the contributors, nor their associated universities 
%      or organizations may be used to endorse or promote products derived from
%      this software without specific prior written permission.
% 
% THIS SOFTWARE IS PROVIDED BY THE COPYRIGHT HOLDERS AND CONTRIBUTORS "AS IS"
% AND ANY EXPRESS OR IMPLIED WARRANTIES, INCLUDING, BUT NOT LIMITED TO, THE
% IMPLIED WARRANTIES OF MERCHANTABILITY AND FITNESS FOR A PARTICULAR PURPOSE ARE
% DISCLAIMED. IN NO EVENT SHALL THE COPYRIGHT OWNER OR CONTRIBUTORS BE LIABLE
% FOR ANY DIRECT, INDIRECT, INCIDENTAL, SPECIAL, EXEMPLARY, OR CONSEQUENTIAL
% DAMAGES (INCLUDING, BUT NOT LIMITED TO, PROCUREMENT OF SUBSTITUTE GOODS OR
% SERVICES; LOSS OF USE, DATA, OR PROFITS; OR BUSINESS INTERRUPTION) HOWEVER
% CAUSED AND ON ANY THEORY OF LIABILITY, WHETHER IN CONTRACT, STRICT LIABILITY,
% OR TORT (INCLUDING NEGLIGENCE OR OTHERWISE) ARISING IN ANY WAY OUT OF THE
% USE OF THIS SOFTWARE, EVEN IF ADVISED OF THE POSSIBILITY OF SUCH DAMAGE.
% 
% ----END-OF-LEGAL-STUFF----
%% Note---the faq:* labels do not get parsed (so that this section
%% stays as one html page in html format), so references to the faq
%% should be directed to sec:faq, not subsections.

\section{Frequently Asked Questions}
\label{sec:faq}

\subsection{Registration UIDs}
\label{faq:reguid-faq}

\subsubsection*{I have a question about \typename{registartion\_uid\_t}(s) 
\ldots basically when do I need them and where do I get them from?}


Registration UIDs within Mace services are mainly handled by the
compiler.  When, in a \mac file, you do:

\begin{programlisting}
downcall_route(dest, msg(some, stuff), router_);
\end{programlisting}

\variablename{router\_} is the registration uid for the Route service 
named \variablename{router\_} in the \symbolkw{services} block.
Similarly, if you specify that in defaults for the message (e.g.
  
\begin{programlisting}
join_msg [downcall_route(const MaceKey&, const Message&, registration_uid_t regId = router_);] {
   //stuff
}
\end{programlisting}

\variablename{router\_} is still the \typename{registration\_uid\_t}.

However, when doing upcalls from the \mac file to a higher level (i.e. to the
application), you need to know the \typename{registration\_uid\_t} of the
higher level.

For example, when the application calls \function{query(\ldots)} on your Mace
service, the last parameter of the query call will be the
\typename{registration\_uid\_t} of the registered
\classname{QueryResponseHandler} to which the response should go.

Typically, You can implement this by including the
\typename{registration\_uid\_t} from the query call in the messages and
responses.  Then, when doing the upcall -- pass in that
\typename{registration\_uid\_t} from the message.  The supporting code
generated by the Mace compiler will take care of looking up the correct handler
to make the call to.

From the application -- after you create the service, you need to
register the query response handler.  If you pass in \literal{-1}, the
service will return you a registration UID.  If you pass in something
else, that will be used as the registration UID.  If you want to
create your own, you can get one from the \classname{NumberGen} class.  But the
first technique is preferable.  Once you have a registration UID in
the application -- you should use that for all further calls to that
service (i.e. future registrations, calls for queries, etc.)

Note: Alternately from the application, you may call \function{registerUniqueHandler}
on the service, as in the first \literal{Ping} application.  This has the effect of
registering your handler with a default and well known UID.  Then, when
making future calls, you may omit the last parameter, allowing it 
the [well known] default value for the UID.  Note that this only works
if a single application calls \function{registerUniqueHandler}, as 
implied by its name (and the fact that only one handler can be registered
under any given UID).

\subsection{Compiler Errors}
\label{faq:compiler_errors}

TBD

\subsection{Compiler Warnings}
\label{faq:compiler_warnings}

TBD

\subsection{Linker Errors}
\label{faq:linker_errors}

\subsubsection*{When I try to link my application, I get a message such
as ``undefined reference to\\
`\replaceable{Myservice}\_namespace::new\_\replaceable{Myservice}\_\replaceable{ServiceClass}'
''.  Why can't it find it?}

The straightforward answer, and the one most useful if the suggestions
below don't help, is that the linker cannot find the library your
service is available in.  Each service directory is compiled into one or
more libraries for of the format
\replaceable{dirname}.\replaceable{compiletype}.a, depending on the
compile type you're using (typically O2 or O0, for optimization levels),
assuming you haven't touched the Makefiles for the service directory.
When you link your application, you need to make sure that you include
each of the service directory libraries you will be using.  If more than
one service is contained in a single directory, all services will be
included in that directory's library.  The library itself is generated
in the services directory, so you can see if it exists there.  If not,
you may just need to run \command{make} at a high enough level to build
the services directory.  Another problem occurs with the order of the
linked directories.  \command{g++} is not smart enough to chase down all
dependencies, so you may need to include libraries more than once.  In
general, you solve this problem by in
\filename{services/Makefile.services} by listing the services in order
such that future directories depend on earlier directories.  The default
application makefile then generates the \variablename{RSERVICES}
variable which is the reverse order, the one appropriate for linking.
Recently another problem was observed by a mace user---setting
\variablename{LIBNAME} in the application makefile to be the same name
as a directory in services caused the service directory to not be found,
instead linking in the application library twice.  So make sure your
application \variablename{LIBNAME} does not collide with any other
libraries you use, including directory names for services.

%{{{
\begin{comment}
\subsection{``No Service registered with \ldots '' error}
\label{faq:servinit_faq}

\subsubsection*{When running an application I get a ``No Service registered
with \ldots'' error.  Do I have to do anything special to
register the service?}


This error is being generated by the service initializer, and implies
that you have asked it to create an instance of a service registered
for a given service class, and no such service has been registered.

This can happen for many reasons:

\begin{itemize}
\item Service is not loaded: A service \literal{Foo} loads itself with the
  service initializer for all service classes it has in its provides
  statement when \function{Foo\_load\_protocol()} is called (in the global
  namespace).  So the load function may not have been called.
\item Service doesn't `Provide' the interface you're using: The service class
  must appear in the provides block to get registered.  Saying that a service
  provides \classname{OverlayRouter} will only register it for
  \classname{OverlayRouterServiceClass*}, not for \classname{RouteServiceClass*},
  which it implements by inheritance. To have both loaded in the service
  initializer, both must be listed on the provides line.
\item Namespace issues: Including header files within namespaces can 
  cause problems with the service initializer.  In particular, since
  the service initializer is templated on the service class type,
  service classes which are included in namespaces will appear
  differently than ones in the global namespace.
\end{itemize}
\end{comment}
%}}}

\subsection{MapIterators, NodeCollections, and Iteration}
\label{faq:nodecollection}

\subsubsection*{Okay, so I'm using \classname{NodeCollection}(s) of \symbolkw{node}-enabled
\symbolkw{auto\_type}(s).  But how do I iterate over all the elements?}

\classname{NodeCollection}(s) provide two methods of iteration. The first is to use the 
\classname{MapIterator} interface.  Suppose for example that you had an \symbolkw{auto\_type} called
\variablename{child}.  Then, in the \symbolkw{typedefs} block, you could have:

\begin{programlisting}
typedefs {
  typedef NodeCollection<child, MAX_CHILDREN> children;
}
\end{programlisting}
This would create a convenient type named \variablename{children} to
refer to these node collections.  Then, in your state variables you
could define a children variable like this:

\begin{programlisting}
state_variables {
  children myKids;
}
\end{programlisting}

Then, to iterate over myKids, you could do so using the
\classname{MapIterator} either in a \symbolkw{for} loop or a
\symbolkw{while} loop.  You'll notice, however, that the
\symbolkw{for} loop is a bit non-standard in that the incrementation
portion of the loop is empty.

For:

\begin{programlisting}
for(children::map_iterator i = myKids.mapIterator(); i.hasNext(); )
{
  child& kid = i.next();
  maceout << "Now considering: " << kid.getId() << Log::endl;
}
\end{programlisting}

While:

\begin{programlisting}
children::map_iterator i = myKids.mapIterator();
while(i.hasNext()) 
{
  child& kid = i.next();
  maceout << "Now considering: " << kid.getId() << Log::endl;
}
\end{programlisting}

The second technique for iterating over a node collection is to use the
\typename{set\_iterator} type (as returned by \function{setBegin()} and
\function{setEnd()}), which works like standard STL iteration, with the proviso
that its like iterating over the node collection as a set, not as a map (for an
iterator \variablename{i}, \variablename{*i} is the node type the collection is
templated on).

\subsubsection*{I did iteration just like the first technique you suggested over a collection in a message, but the
gnu-c compiler won't compile the code.  What's more, the error message is some
template stuff I don't understand.  Do you know what's wrong?  Or, I'm doing
STL iteration like I see it being done elsewhere, but things aren't compiling.}

You should ask yourself if the collection (i.e. \classname{NodeCollection}, 
\typename{hash\_map}, etc.) is a \symbolkw{const} collection.  If so, you'll 
need to use either the \typename{const\_map\_iterator}
or the \typename{const\_iterator}.  This will allow you 
to iterate, but the items you are iterating over will be constant to prevent
you from changing them.  Also keep in mind that message fields are constant, 
even if the message itself is not -- so anything in a message will
automatically be constant.

\subsection{downcall\_ and upcall\_ helper functions}
\label{faq:helperfns}

\subsubsection*{Since I know the API of the services I'm using and the handlers I can make
calls to, why can't I just use syntax more like making calls on the object?
Why do I have to use ``silly'' \symbolkw{downcall\_} and \symbolkw{upcall\_} helper
functions?}

In theory this could work.  However, since the generated code is handling these
references internally for you, all you have accessible are the registration UIDs.
On a technical note, we want you to use the helper functions, both to be consistent
with the necessity to do so when default values are to be used when making calls or
when serialization instructions have been given, and because having a helper function
gives us (the compiler writers) an ideal place to instrument and insert code to do 
other useful stuff.  One nice thing about this approach is that you can specify default
registration UIDs for each call, and then never have to worry about the references
or the registration UIDs again.  

If the demand is great enough, we have also talked about adding syntax which would let you 
write code to call functions on registration UIDs, which could get mapped appropriately, 
but thus far is unsupported.

%{{{
\begin{comment}
\subsection{Priorities for routing services}
\label{faq:priority}

\subsubsection*{Does the \literal{TCP} Service actually \emph{do anything} with the priority information
provided in the API call?  Why is it part of the call?}

No.  The \literal{TCP} service does not use the priority parameter from the API call.
The priority parameter is part of the \literal{TCP} service route because it is part of
the \classname{RouteServiceClass} API.  And it is part of the
\classname{RouteServiceClass} API because it represents information a routing
service \emph{might} use to determine how to handle the message.  There are no
requirements for \emph{how} a route service treats different priority values,
but they are there to make it possible.  It is therefore recommended that
service designers use appropriate priority values to express their desires to
the route service they are using.

For examples of services which do or will use the priority value, look to the
\literal{Pastry} service.  It does not make [much] sense to create multiple \literal{Pastry}
services the way it does with \literal{TCP} services due to the overhead of maintaining a
\literal{Pastry} overlay.  However, it is still desirable to have a way to put messages
in different queues.  In fact, the \literal{Pastry} implementation maps priority values
to different \literal{TCP} instances it uses for routing data.  Another service which
would use the priority value if its ever written is a priority-queue reliable
transport service based over \literal{UDP}.  This service will always focus on sending
the message with the most priority, which may cause it to displace lower
priority messages.
\end{comment}
%}}}

\subsection{64-bit Mace}

\subsubsection*{Can I run Mace unmodified on 64-bit systems?}

Short answer, no.

We would like to get this working, but we are researchers rather than
full-time developers, and it's enough work keeping everything running on
32-bit systems.  Furthermore, we just don't have 64-bit systems to
test/run the system on, so we wouldn't be able to test it anyway.  We do
welcome tips and patches which make progress towards the goal of being
64-bit clean.  Additionally, we were recently tipped about a work-around
if you want to run Mace and can live with a 32-bit binary on 64-bit
machines.

What follows is an \emph{unsupported} set of tips from a user on
compiling 32-bit Mace binaries on and for x86\_64 systems.

\begin{description}
\item[\filename{Makefile.vars}] added \literal{-m32} to \variablename{GLOBAL\_VARS}
\item[\filename{compiler/Makefile}] added -m32 to \variablename{CFLAGS}
and \variablename{CXXFLAGS}, and rules for \command{macecc} and
\command{serviceparser} 
\item[\filename{compiler/XmlRpc/Makefile}] added -m32 to
\command{xmlrpcc} rule
\item[\filename{application/Makefile.common}] added -m32 to the
\literal{\$(APPS)} rule. Added another -lcrypto to the
\literal{LIBS+=rule} of \variablename{MACE\_EXTRAS\_SHA1}.
\item[copied files] \filename{/usr/include/openssl/opensslconf-i386.h}
to the top mace directory, \filename{/usr/lib/libl.a} to \directory{lib} directory, and
\filename{/usr/lib/libcrypto.a} to the \directory{lib} directory.
\end{description}

A few extra comments from the tip:
\begin{verbatim}
Mace is, unfortunately, not 64-bit safe. But apparently, RHEL 64-bit can
compile and run 32-bit code (although some of the 32-bit libraries are
missing, so you have to grab them from a 32-bit machine).
\end{verbatim}

\subsection{On the Mace Library}

\subsubsection*{Is \function{KeyRange::containsKey} a mace-specific
thing or a C++ thing that I don't know?  If it's mace-specific, where do
I find the implementation (so that I know which order the args come
in.)}

Look in \filename{lib/KeyRange.h} for definitions.

Note that \classname{KeyRange} is left/start inclusive, and right/end
exclusive.  $\mathrm{start}==\mathrm{end}$ is interpreted as the entire
range, not the empty range.

As a special note for Chord, typical Chord usages are left-exclusive and
right-inclusive.  (A Chord node address space is from right after your
predecessor to and including its own id).  On the other hand, the
implemented behavior is the standard Pastry/Bamboo use.   In the Mace
implementation of Chord, to handle this there is a lot of stuff based on
``plusone''.


\subsubsection*{Where is \typename{mace::deque} defined?  I tried
\filename{lib/collections.h}, and I \command{grep}ed \directory{mace/lib}
but didn't have much luck.}

Your \command{grep} probably failed because you looked for
\literal{mace::deque}.  The \literal{mace::} part is a namespace, so it
doesn't appear in the definition.  The code would look something like:

\begin{programlisting}
namespace mace {
  class deque {

  };
}
\end{programlisting}

So if you were to \command{grep} for \literal{class deque}, you would
find it much easier.  Unfortunately, it's hard to tell whether the
\literal{::} imply (1) a static method of a class, (2) an inner class,
or (3) a namespace.

Anyway, \literal{mace} is a namespace, and the `collections' defined in
the mace namespace are for the most part just extensions of C++ STL
counterparts.  So \typename{mace::deque} extends from
\typename{std::deque}.  All of these collections are defined in files of
the form \filename{m}\replaceable{foo.h} or
\filename{m\_}\replaceable{foo.h} in the \directory{lib} directory.
Generally we use the underscore when \replaceable{foo} begins with
\literal{m}.  So you'd find \filename{mdeque.h} or \filename{m\_map.h} in
the \directory{lib} directory.

If you're looking for documentation on the methods, for the most part
you'll want to see instead the documentation of the STL parts.  A good
reference here is \href{http://www.sgi.com/tech/stl/}{the SGI webpage
[http://www.sgi.com/tech/stl/]}.

Some systems have the STL manpages installed, so then you can just do
\command{man std::deque}.

The reason Mace has counterparts of these STL things is to provide
print-ability and serialize-ability.  We also add a few convenience
methods like \function{containsKey} (for maps), though these are often
unused for efficiency.

Oh, as a final note, one thing you might should know is that
constructors are not inherited.  We've only defined the ones we need
usually.  If you need another one, they are easy to add.  The defined
ones will appear in the mace \filename{.h} file.

\subsubsection*{The STL documentation says that you shouldn't inherit
from STL collections.  Does that mean Mace code is broken or unsafe?}

STL classes were defined without a virtual destructor.  This is a
mechanism they use to discourage inheriting from them.  In our (perhaps
incomplete) understanding of this mechanism, the only cause for concern
is if you were to delete a pointer to the base class type when the
object was implemented by an inherited class.  In our code we avoid the
use of pointers, and generally only use the mace template types anyway,
so we believe that this use is relatively safe.  Without the inherited
classes, we would have to come up with complicated ways to add new
methods to STL types, or define new collection types.  Since this method
seems mostly safe and it was by far the easiest method to get it
working, we went ahead and inherited from them.  So now you know what to
watch our for.  Be careful!

